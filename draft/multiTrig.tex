\documentclass{amsart}

%%%%%%%%%%%%%%%%%%%%%%%%%%%%%%%%%%%%%%

\usepackage[utf8]{inputenc}
\usepackage[T1]{fontenc}

\usepackage{a4wide}%en grand
\usepackage{changepage}%indentation

\usepackage{xcolor}
\usepackage{amsfonts,amsthm,amssymb,amsmath}
\usepackage{mathtools}
\usepackage{wasysym}
\usepackage{xspace}
\usepackage{graphicx}
%\usepackage[notcite,notref]{showkeys} % shows labels 
\usepackage{breqn}
\usepackage{algorithm}
\usepackage{algorithmic}
\usepackage{tabularx}

\usepackage[english]{babel} %gestion des langues
\usepackage{caption}
\usepackage{subcaption}
\usepackage{paralist}
\usepackage{multirow}

\usepackage{hyperref}
\hypersetup{colorlinks=true, citecolor=darkblue, linkcolor=darkblue}

\usepackage[noabbrev,capitalise]{cleveref}
\usepackage{autonum}
\usepackage{xspace}

%%%%%%%%%%%%%%%%%%%%%%%%%%%%%%%%%%%%%%

\newtheorem{theorem}{Theorem}[section]
\newtheorem{proposition}[theorem]{Proposition}
\newtheorem{lemma}[theorem]{Lemma}
\newtheorem{ce}[theorem]{Counter-example}
\newtheorem{claim}[theorem]{Claim}
\newtheorem{corollary}[theorem]{Corollary}
\newtheorem{definition}[theorem]{Definition}
\newtheorem{notation}[theorem]{Notation}
\theoremstyle{remark}
\newtheorem{remark}{Remark}[section]
\newtheorem{example}{Example}
\newtheorem{algo}{Algorithm}
\newtheorem*{example*}{Example}

\crefname{theorem}{Theorem}{Theorems}
\crefname{lemma}{Lemma}{Lemmas}

\definecolor{darkblue}{rgb}{0,0,0.7} % darkblue color
\newcommand{\darkblue}{\color{darkblue}} % darkblue command
\newcommand{\defn}[1]{\textsl{\darkblue #1}} % emphasis of a definition

%%%%%%%%%%%%%%%%%%%%%%%%%%%%%%%%%%%%%%

\newcommand*{\dual}[1]{{#1^*}}
\newcommand*{\nbd}[0]{neighbourhood\xspace}
\newcommand*{\ef}[0]{E-finite\xspace}
\newcommand*{\vf}[0]{V-finite\xspace}
\newcommand*{\ktg}[0]{$k$-triangulation\xspace}

\graphicspath{{../images/}}

%%%%%%%%%%%%%%%%%%%%%%%%%%%%%%%%%%%%%%

\title[Infinite multitriangulations and multitriangulations of surfaces]{Infinite multitriangulations \\ and multitriangulations of surfaces}

\thanks{ML was partially supported by the French ANR grant GATO~(16\,CE40\,0009). \\ \indent VP was partially supported by the French ANR grants SC3A~(15\,CE40\,0004\,01) and CAPPS~(17\,CE40\,0018).}

\author{Mathias Lepoutre}
\address{LIX, \'Ecole Polytechnique, Palaiseau}
\email{mathias.lepoutre@lix.polytechnique.fr}
\urladdr{\url{http://www.lix.polytechnique.fr/Labo/Mathias.Lepoutre/}}

\author{Vincent Pilaud}
\address{CNRS \& LIX, \'Ecole Polytechnique, Palaiseau}
\email{vincent.pilaud@lix.polytechnique.fr}
\urladdr{\url{http://www.lix.polytechnique.fr/~pilaud/}}

%%%%%%%%%%%%%%%%%%%%%%%%%%%%%%%%%%%%%%

\begin{document}

\begin{abstract}
We extend previous work on the structure of $k$-stars of a multi-triangulation on a convex polygon to the case of multi-triangulations on any surface, orientable or not. 

To that extent, we use the universal cover construction, that makes a map on a surface into a periodic map of an infinite polygon. We generalize the work of Pilaud and Santos to multi-triangulation of an infinite polygon, with some additional constraints, and then conclude about the case of multi-triangulations on any surface.
\end{abstract}

\maketitle

\section{Definitions}


\begin{definition}[infinite polygon]
An \defn{infinite polygon} is a circle with an infinite but ordered set of points on it.
\end{definition}

From now on, the word polygon will be used equally for both finite and infinite polygons.

\begin{definition}[left/right neighbour, \nbd]
If $x$ and $y$ are two points of a polygon, such that the interval $[x,y]$ is empty, we say that $y$ is the \defn{right neighbour} of $x$, while $x$ is the \defn{left neighbour} of $y$.

A polygon is a \nbd if each point has both a right and a left neighbour.
\end{definition}

In particular, any finite polygon is a \nbd.

\begin{definition}[$k$-triangulation]
A $k$-triangulation of a polygon $P$ is a maximal set of edges of the points of $P$ with no $(k+1)$-crossing.
\end{definition}

\begin{definition}[\ef, \vf]
A $k$-triangulation is \ef if any of its edge is crossed be a finite number of edges.

A $k$-triangulation is \vf if all its vertices have finite degree.
\end{definition}

\begin{definition}[periodic]

\end{definition}


\begin{theorem}\label{thm:flip}
Any edge of a \ef \ktg of a \nbd can be flipped.
\end{theorem}

\section{Proof of \cref{thm:flip}}

\subsection{crossings crossing an angle}

La formulation exacte des lemmes devra être vérifiée au moment d'en écrire les preuves.

\begin{lemma}
Any angle of a \ktg of a \nbd is crossed by a $(k-1)$-crossing.
\end{lemma}
\begin{proof}
The $k$-border always contains such a crossing.
\end{proof}


\begin{lemma}
The set of $(k-1)$-crossings crossing a given angle of a \ktg forms a distributive lattice.
\end{lemma}
\begin{proof}
Let $AB=\{(a_1,b_1),\cdots,(a_{k-1},b_{k-1})\}$ and $CD=\{(c_1,d_1),\cdots,(c_{k-1},d_{k-1})\}$ be two $(k-1)$-crossings crossing the angle $(u,v,w)$.

There is no $i$ such that $(a_i,b_i)$ crosses $(c_i,d_i)$. 
Indeed, suppose $a_i<c_i<v<b_i<d_i$. Then $\{(u,v),(a_1,b_1),\cdots,(a_i,b_i),(c_i,d_i),\cdots,(c_{k-1},d_{k-1})\}$ forms a $(k+1)$-crossings. 
Similarly, if $c_i<a_i<v<d_i<b_i$, then $\{(u,v),(c_1,d_1),\cdots,(c_i,d_i),(a_i,b_i),\cdots,(a_{k-1},b_{k-1})\}$ forms a $(k+1)$-crossings.
Hence $AB$ and $CD$ are edgewise comparable.

We define the meet (resp. join) of $AB$ and $CD$ as their edgewise minimum (resp. maximum). With this definition it is easy to see that we obtain a distributive lattice.
\end{proof}

\begin{lemma}
Any angle of a \ef \ktg which is crossed by a $(k-1)$-crossing, is maximally crossed by a $(k-1)$-crossing.
\end{lemma}
\begin{proof}
Since the \ktg is \ef, the angle is crossed by a finite number of $(k-1)$-crossings. Hence the distributive lattice of such crossings is finite, and has a maximum.
\end{proof}

\subsection{any angle belongs to a $k$-star}

\begin{lemma}
Any angle of a \ktg which is maximally crossed by a $(k-1)$-crossing belongs to a $k$-star.
\end{lemma}
\begin{proof}
cf thm4.1 of  Pilaud Santos, but to be rewritten.



Let $\{e_1, \cdots , e{_k-1}\}$ (with $e_i = [a_i, b_i]$) be a $(k - 1)$-crossing intersecting $\wedge(u, v, w)$ and assume that it is $v$-maximal. 
We will prove that the edges $[u, b_1],[a_1, b_2]\cdots[a_{k-2}, b_{k-1}], [a_{k-1}, w]$ are in $T$ such that the points $u$, $a_1,\cdots,a_{k-1}$, $v$, $b_1,\cdots,b_{k-1}$, $w$ are the vertices of a $k$-star of $T$ containing the angle $\wedge(u, v, w)$. 

To get this result, we use two steps: first we prove that $\wedge(a_1, b_1, u)$ is an angle of $T$, and then we prove that the edges $e_2,\cdots, e_{k-1}, [v, w]$ form a $(k-1)$-crossing intersecting $\wedge(a_1, b_1, u)$ and $b_1$-maximal (so that we can reiterate the argument).

{\bf First step.}
Suppose that $[u, b_1]$ is not in $T$. 
Thus we have a $k$-crossing $F$ that prevents the edge $[u, b1]$.
Let $f_1 = [c_1, d_1],\cdots, f_k = [c_k, d_k]$ denote its
edges with $u < c_1 < \cdots < c_k < b_1$ and $b_1 < d_1 < \cdots < d_k < u$.

Note first that $v < d_k \leq w$. Indeed, if it is not the case, then
$d_k \in ]w, u[$ and $c_k \neq v$, because $\wedge(u, v, w)$ is an angle. 
Thus either $c_k \in ]u, v[$ and then $F \cup \{[u, v]\}$ forms a $(k + 1)$-crossing, 
or $c_k \in ]v, b_1[$ and then $E \cup \{[c_k, d_k]\}$ forms a $(k + 1)$-crossing. 
Consequently, we have $b_1 < d_1 < \cdots < d_{k-1} < w$.

Let $\ell = \text{max}\{1 \leq j \leq k - 1 | b_i < d_i < w, \forall i \setminus 1 \leq i \leq j\}$.
Then for any $1 \leq i \leq \ell$, since $\{e_1, \cdots , e_i\} \cup \{f_i, \cdots , f_k\}$ does not form
a $(k + 1)$-crossing, we have $u < c_i \leq a_i$. Thus for any $1 \leq i \leq \ell$, $u < c_i \leq a_i < v < b_i < d_i < w$, so that $f_i$
is $v$-farther than $e_i$.
Furthermore, we have $u < c_1 < \cdots < c_\ell < a_{\ell+1} < \cdots < a_{k-1} < v < d_1 < \cdots < d_\ell < b_{\ell+1} < \cdots < b_{k-1} < w$. 
Consequently, we get a $(k - 1)$-crossing $\{f_1, \cdots , f_\ell
, e_{\ell+1}, \cdots , e_{k-1}\}$ which is $v$-farther than $\{e_1, \cdots , e_{k-1}\}$; this contradicts the definition of $\{e_1, \cdots , e_{k-1}\}$. 
Thus we obtain $[u, b_1] \in  T$.

Suppose now that $\wedge(a_1, b_1, u)$ is not an angle of $T$. 
Then there exists $a_0 \in ]u, a_1[$ such that $[b_1, a_0] \in  T$. 
But then the $(k - 1)$-crossing $\{[a_0, b_1], e_2, \cdots , e_{k-1}\}$ is $v$-farther than $\{e_1, \cdots , e_{k-1}\}$. 
This implies that $\wedge(a_1, b_1, u)$ is an angle of $T$.

{\bf Second step.}
Let $F$ be a $(k - 1)$-crossing intersecting $\wedge(a_1, b_1, u)$ and $b_1$-farther than the $(k-1)$-crossing $\{e_2, \cdots , e_{k-1}, [v, w]\}$. 
Let $f_2 = [c_2, d_2], \cdots , f_k = [c_k, d_k]$ denote its edges, with $a_1 < c_2 < \cdots < c_k < b_1 < d_2 < \cdots < d_k < u$.

Note first that $b_k \leq d_k \leq w$. 
Indeed, if it is not the case, then $d_k \in ]w, u[$ and $c_k \neq v$, because $\wedge(u, v, w)$ is an angle. 
Thus either $c_k \in ]a_1, v[$ and then $F \cup \{[u, v], e_1\}$ forms a $(k + 1)$-crossing, or $c_k \in ]v, b_1[$ and then $E \cup \{[c_k, d_k]\}$ forms a $(k + 1)$-crossing. 
Consequently, we have $b_1 < d_2 < \cdots < d_{k-1} < w$.

Furthermore, for any $2 \leq i \leq k - 1$, $f_i$ is $\wedge(a_1, b_1, u)$-farther than $e_i$, so that $a_1 < c_i \leq a_i < b_1 < b_i \leq d_i < u$. 
In particular, $a_1 < c_{k-1} \leq a_{k-1} < v$ and we get $u < a_1 < c_2 < \cdots < c_{k-1} < v < b_1 < d_2 < \cdots < d_{k-1} < w$.
Consequently, the $(k - 1)$-crossing $\{e_1, f_2, \cdots , f_{k-1}\}$ is
$v$-farther than $\{e_1, \cdots , e_{k-1}\}$, which is a contradiction.













\end{proof}

We may need the \ktg to be \vf?

\subsection{any edge belongs to $2$ $k$-stars, and can be flipped}

\begin{lemma}
When $k\geq 2$, any \ef \ktg of a \nbd is \vf.
\end{lemma}
\begin{proof}
Any edge of the $k$-border surrounding a given vertex intersects all edges adjacent to this vertex.
\end{proof}

\begin{lemma}
Any edge of a \vf \ktg belongs to $4$ angles.
\end{lemma}
\begin{proof}
The vertices adjacent to the edge have finite degree. Take the neighbours just before and after the other vertex. The $4$ hereby obtained edges form angles with the first one.
\end{proof}

\begin{lemma}
Any edge adjacent to $2$ $k$-star can be flipped.
\end{lemma}
\begin{proof}
cf Pilaud Santos. to be adapted to the infinite case, but shouldn't need any additional hypothesis. quite long though : section 3.
\end{proof}

\section{More}

\subsection{more lemmas}

\begin{lemma} 
Any edge of a \ef \ktg of a \nbd belongs to 4 angles. 
\end{lemma}
\begin{proof}
see direct proof on sheet.
\end{proof}

\subsection{some counter-examples}

\begin{ce}
The polygon of a periodic \ktg is not necessarily a \nbd.
\end{ce}
\begin{proof}

\end{proof}

\begin{ce}
A periodic \ef \ktg may have angles that are not crossed by a $(k-1)$-crossing.
\end{ce}
\begin{proof}

\end{proof}

\begin{ce} 
A periodic \ktg may not be \ef.
\end{ce}
\begin{proof}

\end{proof}

\begin{ce}
A \ef \ktg of a \nbd is not necessarily \vf.
\end{ce}
\begin{proof}
k=1
\end{proof}




\bibliographystyle{abbrv}
\bibliography{../bibliography}

\end{document}
